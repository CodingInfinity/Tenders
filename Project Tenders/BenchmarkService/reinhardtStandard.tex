%%%%%%%%%%%%%%%%%%%%%%%%%%%%%%%%%%%%%%%%%
% "ModernCV" CV and Cover Letter
% LaTeX Template
% Version 1.11 (19/6/14)
%
% This template has been downloaded from:
% http://www.LaTeXTemplates.com
%
% Original author:
% Xavier Danaux (xdanaux@gmail.com)
%
% License:
% CC BY-NC-SA 3.0 (http://creativecommons.org/licenses/by-nc-sa/3.0/)
%
% Important note:
% This template requires the moderncv.cls and .sty files to be in the same 
% directory as this .tex file. These files provide the resume style and themes 
% used for structuring the document.
%
%%%%%%%%%%%%%%%%%%%%%%%%%%%%%%%%%%%%%%%%%

%----------------------------------------------------------------------------------------
%   PACKAGES AND OTHER DOCUMENT CONFIGURATIONS
%----------------------------------------------------------------------------------------

\documentclass[11pt,a4paper,sans]{moderncv} % Font sizes: 10, 11, or 12; paper sizes: a4paper, letterpaper, a5paper, legalpaper, executivepaper or landscape; font families: sans or roman
\usepackage[utf8]{inputenc}
\moderncvstyle{casual} % CV theme - options include: 'casual' (default), 'classic', 'oldstyle' and 'banking'
\moderncvcolor{blue} % CV color - options include: 'blue' (default), 'orange', 'green', 'red', 'purple', 'grey' and 'black'
\usepackage{lipsum} % Used for inserting dummy 'Lorem ipsum' text into the template

\usepackage[scale=0.75]{geometry} % Reduce document margins
%\setlength{\hintscolumnwidth}{3cm} % Uncomment to change the width of the dates column
%\setlength{\makecvtitlenamewidth}{10cm} % For the 'classic' style, uncomment to adjust the width of the space allocated to your name

%----------------------------------------------------------------------------------------
%   NAME AND CONTACT INFORMATION SECTION
%----------------------------------------------------------------------------------------

\firstname{Reinhardt} % Your first name
\familyname{Cromhout} % Your last name

% All information in this block is optional, comment out any lines you don't need
\title{Team Member}
\address{1284 Prospect Street}{Hatfield, South Africa}
\mobile{+2782 301 7299 }


\email{reinhardt.cromhout@gmail.com}
\photo[60pt][0.4pt]{../images/reinhardt.jpg} % The first bracket is the picture height, the second is the thickness of the frame around the picture (0pt for no frame)

%----------------------------------------------------------------------------------------

\begin{document}

\makecvtitle % Print the CV title

%----------------------------------------------------------------------------------------
%   EDUCATION SECTION
%----------------------------------------------------------------------------------------

\section{Education}
\cventry{2009-2013}{Secondary Education}{Helpmekaar Kollege}{tepatitlan}{IEB Matric Sertificate - 7 A's}{}{}

\cventry{2014-2016}{BSc Computer Science (Still in final year of Degree)}{University iof Pretoria}{}{Majoring in Computer Science}{}{}

\section{Experience}
\subsection{Vocational}
\cventry{2015}{Temporary Part Time Technical Assistant}{University of Pretoria}{Systems adminstration for the Department of Information Science}{}{}
%----------------------------------------------------------------------------------------
%   AWARDS SECTION
%----------------------------------------------------------------------------------------

\section{Awards}

\cvitem{2014}{Best first year Computer Science Student at the Uiverstity of Pretoria}

%----------------------------------------------------------------------------------------
%   COMPUTER SKILLS SECTION
%----------------------------------------------------------------------------------------

\section{Skills }

\cvitem{Basic}{URDAD mocking framework, Microsoft Sharepoint Server 2013, Microsoft SQL Database Server 2008, Microsoft Active Direcory
Server(server administration), Hosting the LAMP Stack, Apache Maven}
\cvitem{Intermediate}{mySQL, Bootstrap, php(as well as connecting php and mySQL),AJAX, Java EE and Spring dependancy injection, JUnit, git, git-flow(source version control)}
\cvitem{Expert}{Java, C++, MS SQL 2008,HTML5, CSS3, JavaScript/jQuery}
%----------------------------------------------------------------------------------------
%   LANGUAGES SECTION
%----------------------------------------------------------------------------------------

\section{Languages}

\cvitemwithcomment{English}{Basic}{Technical reading }

%----------------------------------------------------------------------------------------
%   INTERESTS SECTION
%----------------------------------------------------------------------------------------

\section{Interests}
\renewcommand{\listitemsymbol}{-~} % Changes the symbol used for lists
\cvlistdoubleitem{Tennis, Cricket, Rugby}{Learning new Development Technologies}
\cvlistdoubleitem{Music Festivals}{Gymnasium exercise}

\section{About Me}
As a third year Computer Science student at the University of Pretoria I have excelled at my first and second
year subjects. I received an award as the top BSc Computer Science student in 2014 and the second best BSc
Computer Science student in 2015. I have always focused on gaining useful knowledge and mastering con-
cepts and the good marks are just a result of that. I worked for a year as a systems administrator for the School
of IT at the University of Pretoria.
Due to the concept based approach of my degree I am not only proficient in the languages and technolo-
gies that I have been taught, but I have the ability to learn similar languages and technologies fairly quickly.
In addition I also have strong people skills and am comfortable with leading meetings, delegating tasks and
enforcing deadlines.

\section{Relevant Past Experience}
Aside from passing all my first and second year subjects with a general average of 85\%,
I worked for a year in 2015 as a "Temporary Part Time Technical Assistant" for the Department of Information Science
at the University of Pretoria.

\section{Why I want to do the project}
I feel that this is a gap in the market. There are many people from researchers to workers in industry who would benefit from
a benchmarking tool that is generic and easy to use in the sense that it does not require complex configuration. This project is both
challenging and useful in the real world.

\end{document}
